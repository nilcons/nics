% -*- latex -*-

% Short names for commonly required computer science characters
% Tests: test-basic
\def\bs{\textbackslash}
\def\tld{\textasciitilde}

% Use the full page as a 16x9 canvas
\pagewidth 16cm
\pageheight 9cm
\topskip 0pt
\pdfvariable horigin 0pt
\pdfvariable vorigin 0pt
\hoffset 0pt
\voffset 0cm
\parindent 0pt
\parskip 0pt

% Contains the current working directory of the slides.tex file that
% we are building.  Set by nics-slide.lua.
\newtoks\nicspwd

\makeatletter

% By default a section name is used for PDF outline name too.
%
% You can do \section[]{Introduction} and then no PDF outline item will be generated at all for this section.
% Or you can do: \section[PDF outline alternative]{Heavy $x^2$ \TeX\ math that would not be a valid PDF outline item}
%
% The \section[...]{...}
%
% Tests: test-titlepage
\newtoks\@nicstok@section
\newcommand{\section}[2][\relax]{%
  \@nicstok@section{#2}%
  \if\relax\detokenize{#1}\relax%
  \else%
  \@nicsmcr@outlineitem{\if\relax#1#2\else#1\fi}%
  \fi}

% The slide environment is used to represent one slide in the presentation.
% Nics doesn't support frames-and-slides (as Beamer), this is an intentional simplification.
% Nics renders every slide into a box and emits every page at the end, this:
%   - improves speed in face of syntax errors (common case),
%   - only slighlty decreases speed in case of no errors (~2-5%),
%   - makes it possible to have page numbers without double compilation.
%
% Usage: \begin{slide}{title}{subtitle}...\end{slide}
%
% Tests: test-basic, test-titlepage
\newcount\nicspagenumber\nicspagenumber1 % Page numbers enabled by default
\newcount\nicsgrid % A debug grid for positioning
\newcount\nicsfooter\nicsfooter1 % Footers enabled by default
\newcount\nicsallowlocalpars % Can be used to override the mechanism that disables implicit paragraphs
\newcount\@nicscnt@currentslide
\newcount\@nicsbool@nolocalpars
\def\nicseveryslide{}
\newenvironment{slide}[2]{%
  \global\advance\count0 by 1% slide counter is stored in the 0th count register
  \baselineskip0pt%
  \newbox\@nicsbox@nextslide% reserve a new box for the next slide
  \expandafter\@nicscnt@currentslide\the\@nicsbox@nextslide% @nicscnt@currentslide is used in nics-slide.lua
  \global\setbox\@nicsbox@nextslide\vbox\bgroup{% render the new slide and remember the box globally
    \pdfextension dest name{nicsoutlinetarget\romannumeral\count0} xyz\relax% have a PDF target at every page
    \@nicsmcr@trybgpic% emit a background picture if one is set via \nicsbgimg
    \@nicsmcr@trygrid% emit debug grid if enabled
    \nicseveryslide% callback that you can use e.g. for a logo
    \nicstitle{#1}%
    \nicssubtitle{#2}%
    \@nicsmcr@footer% emit footer
  }%
  \@nicsbool@nolocalpars1\relax% disable implicit paragraph starting of TeX
}{%
  \egroup% this ends the \vbox's \bgroup
  \@nicslfn@saveslide% defined in nics-slide.lua
}

\csname @nicsmcr@bgpic\endcsname
\def\nicsbgimg#1{\def\@nicsmcr@bgpic{#1}}

\def\@nicsmcr@trybgpic{%
  \ifx\@nicsmcr@bgpic\relax\else%
  \vbox to 0pt{\directlua{nicsexterninline("bgpic", "", "", "", "\the\nicspwd \@nicsmcr@bgpic")}\vss}%
  \fi}

% Tests: test-titlepage
\def\@nicsmcr@trygrid{%
  \ifnum\nicsgrid=1
  \vbox to 0pt{\includegraphics{\the\nicsroot /grid/grid}\vss}%
  \fi}

\def\@nicsmcr@footer{%
  \ifnum\nicsfooter=1%
  \vbox to 0pt{\kern8.15cm \hbox to 1.1\hsize{\leaders\hbox{\vrule width 5.5pt height 0.3pt\relax\kern0.6pt}\hss}\vss}%
  \vbox to 0pt{\kern8.36cm \hbox to \hsize{\kern3mm \Small \strut\the\@nicstok@section\hss}\vss}%
  \fi%
}

% Tests: test-basic
\def\nicstitle#1{\vbox to 0pt{\kern5mm\noindent\kern5mm{\color{Black}\Large \strut #1}\vss}}
\def\nicssubtitle#1{\vbox to 0pt{\kern11mm\noindent\kern5mm{\color{RoyalBlue}\Small \strut #1}\vss}}

% Tests: test-titlepage
\def\@nicsmcr@centertitle#1#2{\vbox to 0pt{\kern#1\centering{%
      \directlua{nicsexterninline("tcolorbox", "", "", "",
        \luastringN{\tcbox[opacityfill=0.2,boxrule=0pt,colback=black,coltext=white]}
        .. "{" .. \luastringN{#2} .. "}")}}\vss}}

\def\nicsbigtitle#1{\@nicsmcr@centertitle{1.25cm}{\Large \strut#1}}
\def\nicsbigsubtitle#1{\@nicsmcr@centertitle{7.5cm}{\Small \strut#1}}
\def\nicsattributebg#1{\vbox to 0pt{\kern8.5cm{\hbox{\kern5mm{\tiny\lightfont\color{white}\strut#1}}}\vss}}
\def\nicsmultiline#1{\begin{tabular}{c}\strut#1\strut\end{tabular}}

% Params:
%   - optional: background picture attribution
%   - background picture
%   - title
%   - subtitle
% Tests: test-titlepage
\newcommand{\nicstitleslide}[4][]{{
  \nicsfooter0
  \nicspagenumber0
  \nicsbgimg{#2}
  \begin{slide}{}{}
    \nicsbigtitle{#3}
    \nicsbigsubtitle{#4}
    \nicsattributebg{#1}
  \end{slide}
}}

\makeatother

% Try to get rid of all hyphenations, set hsize and default font, page count to 0.
\AtBeginDocument{\hyphenpenalty10000\tolerance10000\relax\hsize16cm\sffamily\count0=0\relax}

% Annoying catcodes off...
\AtBeginDocument{\catcode`\&=12\catcode`\$=12\catcode`\^=12\catcode`\_=12\relax}
% But a way for math people to get them back.
\def\nicsmath{\catcode`\&=4\catcode`\$=3\catcode`\^=7\catcode`\_=8\relax}
