% -*- latex -*-
% Main logic and comments in nics-extern.lua

% Template to use for the next externalization.
\newtoks\nicsexterntemplate\nicsexterntemplate{default}

% Set to 1 if you want nicsextern re-inclusion to be not centered.
\newcount\nicsexternnoautocenter

% Extra stuff to include in the externalization, after the preamble.
\newtoks\nicsexternextra

% Can be used by the end-user to reuse the last externalized part
% with \nicsincludeextern without repeating the name (DRY).
\newtoks\nicsexternlastname

\makeatletter

% Set to 1 if you don't want the generation of \newtoks\nicspwd...,
% currently only needed by \nicsexterncode, can be made public
\newcount\@nicsbool@externnopwd

% Set to 1 if you don't want nicsextern to remove initial whitespace
% from the body.  This is done by default, so you can indent the body
% of the nicsextern environment normally, without having additional
% whitespace in the output.  This is another setting that is currently
% only disabled by \nicsexterncode, but can be made public if needed.
\newcount\@nicsbool@externinitialwhitespace

% Used to store the name of the symlink that we should create for this
% externalization, empty string means to not create a symlink.
\newtoks\@nicstok@externname

% Remembering \includegraphics options between the start and end of
% the nicsextern environment.
\newtoks\@nicstok@externoptions

% Remembering whether we called nicsextern from hmode or vmode.
\newtoks\@nicstok@externmode

% The XXX set by \nicsexterncode{XXX}, passed to code.sh
\newtoks\@nicstok@externshellparam

\def\nicsexterncode#1{%
  \nicsexterntemplate{code}%
  \@nicstok@externshellparam{#1}%
  \@nicsbool@externnopwd 1\relax%
  \@nicsbool@externinitialwhitespace 1\relax%
}


\newenvironment{nicsextern}[2][width=\hsize]{%
  \ifhmode\@nicstok@externmode{h}\else\ifvmode\@nicstok@externmode{v}\fi\fi%
  \bgroup%
    \@nicstok@externoptions{#1}%
    \@nicstok@externname{#2}%
    \@nicslfn@externstart%
}{%
  \egroup%
}

\makeatother
